\documentclass[10pt]{article}

\usepackage[margin=1in]{geometry} 
\usepackage{amsmath,amsthm,amssymb, graphicx, multicol, array}

\newtheorem{innercustomgeneric}{\customgenericname}
\providecommand{\customgenericname}{}
\newcommand{\newcustomtheorem}[2]{%
	\newenvironment{#1}[1]
	{%
		\renewcommand\customgenericname{#2}%
		\renewcommand\theinnercustomgeneric{##1}%
		\innercustomgeneric
	}
	{\endinnercustomgeneric}
}

\newcustomtheorem{customthm}{Theorem}
\newcustomtheorem{customlemma}{Lemma}

\newcommand{\N}{\mathbb{N}}
\newcommand{\Z}{\mathbb{Z}}
\newcommand{\Q}{\mathbb{Q}}
\newcommand{\R}{\mathbb{R}}
\newcommand{\C}{\mathbb{C}}

\newenvironment{problem}[2][Problem]{\begin{trivlist}
		\item[\hskip \labelsep {\bfseries #1}\hskip \labelsep {\bfseries #2.}]}{\end{trivlist}}

\begin{document}
	
	\title{Exercises from Chapter 5}
	\author{Wesley Basener}
	\maketitle
	\begin{problem}{1}
		Suppose $(X, \mathcal{A})$ is a measurable space, $f$ is a real-valued function, and $\{x : f(x) > r\} \in \mathcal{A}$ for each rational number $r$. Prove that $f$ is measurable.
		\begin{proof}
			We need to show that $\{x : f(x) > q\} \in \mathcal{A}$ holds for any irrational $q$. Let $q$ be any irrational number. Define the sequence of rational numbers $a_0, a_1, a_2,...$ such that $a_i > a_j$ for $i > j$ and $\lim a_i = q$. Then, $\cup_0^\infty \{x : f(x) > a_i\}$ is equivalent to $\{x : f(x) > q\}$ and is the countable union of elements in $\mathcal{A}$, hence is an element of $\mathcal{A}$. Therefore, $\{x : f(x) > r\} \in \mathcal{A}$ for all real $r$, meaning $f$ is measurable.
		\end{proof}
	\end{problem}
	
	\begin{problem}{3}
		Suppose $f$ is a measurable function and $f(x) > 0$ for all $x$. Let $g(x) = 1/f(x)$. Prove that $g$ is a measurable function.
		\begin{proof}
			By proposition 5.5, $f$ being measurable implies that $\{x : f(x) \leq r\} \in \mathcal{A}$ for all real $r$. But each set $\{x : f(x) \leq r\}$ is equal to $\{x : g(x) > r\}$. Hence, $\{x : g(x) > r\} \in \mathcal{A}$ for all $r$ and $g$ is measurable.
		\end{proof}
	\end{problem}
	
	\begin{problem}{5}
		If $f : \R \rightarrow \R$ is Lebesgue measurable, prove that there exists a Borel measurable function $g$ such that $f = g$ a.e.
		\begin{proof}
			Since the Lebesgue algebra is the closure of the Borel algebra, any non-Borel measurable Lebesgue set will have measure $0$. Hence, we can construct a function 
			\[ g(x) = \begin{cases} 
				0 & x \in N\\
				f(x) & \text{else}
			\end{cases}
			\]
			Where $N$ is the union of non Borel measurable Lebesgue sets. This function differs from $f$ on a domain of measure $0$. Hence, $f=g$ a.e.
			//TODO make this more rigorous; pretty sure it doesn't work right now 
		\end{proof}
	\end{problem}
	
	\begin{problem}{7}
		Suppose $f : \R \rightarrow \R$ is differentiable at each point. Prove that $f$ and $f'$ are Borel measurable.
		\begin{proof}
			Differentiability implies continuity, hence $f$ is Borel measurable. Recall the definition of the derivative is 
			\begin{align*}
				f'(x) = \lim_{h\rightarrow 0} \frac{f(x+h) + f(x)}{h}
			\end{align*}
			Which is the limit of the sum of measurable functions times a scalar, hence is a measurable function.
		\end{proof}
	\end{problem}
	
	\begin{problem}
		Suppose $f : \R \rightarrow \R$ is Lebesgue measurable and $g : \R \rightarrow \R$ is continuous. Prove that $g \circ f$ is Lebesgue measurable. Is this true if $g$ is Borel measurable instead of continuous? Is this true if $g$ is Lebesgue measurable instead of continuous?
		\begin{proof}
			
		\end{proof}
	\end{problem}
\end{document}