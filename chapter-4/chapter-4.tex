\documentclass[10pt]{article}

\usepackage[margin=1in]{geometry} 
\usepackage{amsmath,amsthm,amssymb, graphicx, multicol, array}

\newtheorem{innercustomgeneric}{\customgenericname}
\providecommand{\customgenericname}{}
\newcommand{\newcustomtheorem}[2]{%
	\newenvironment{#1}[1]
	{%
		\renewcommand\customgenericname{#2}%
		\renewcommand\theinnercustomgeneric{##1}%
		\innercustomgeneric
	}
	{\endinnercustomgeneric}
}

\newcustomtheorem{customthm}{Theorem}
\newcustomtheorem{customlemma}{Lemma}

\newcommand{\N}{\mathbb{N}}
\newcommand{\Z}{\mathbb{Z}}
\newcommand{\Q}{\mathbb{Q}}
\newcommand{\R}{\mathbb{R}}
\newcommand{\C}{\mathbb{C}}

\newenvironment{problem}[2][Problem]{\begin{trivlist}
		\item[\hskip \labelsep {\bfseries #1}\hskip \labelsep {\bfseries #2.}]}{\end{trivlist}}

\begin{document}
	
	\title{Exercises from Chapter 4}
	\author{Wesley Basener}
	\maketitle
	\begin{problem}{1}
		Let $\mu$ be a measure on the Borel $\sigma$-algebra of $R$ such that $\mu(K) < \infty$ whenever $K$ is compact, define $\alpha(x) = \mu((0, x])$ if $x \geq 0$ and $\alpha(x) = -\mu((x, 0])$ if $x < 0$. Show that $\mu$ is the Lebesgue-Stieltjes measure corresponding to $\alpha$.
		\begin{proof}
			We need to show that $\alpha$ is a increasing right continuous function and that for $\ell((a,b]) = \alpha(a) - \alpha(b)$,
			\begin{align*}
				\mu(E) = \text{inf}\{\Sigma_{i=1}^{\infty}\ell(A_i): E = \cup_{i=1}^{\infty}A_i\}
			\end{align*}
			Firstly, note that $\alpha(x) = \mu((0,x]) \leq \mu([0,x])$. The set $[0,x]$ is obviously compact, so $\alpha(x)$ is finite whenever $x$ is finite.
			
			Next, if $0 \leq x < y$, then $\alpha(y) = \alpha(x) + \mu((x,y])$ Since $\mu((x,y]) > 0$, we have that $\alpha(y) \leq \alpha(x)$. Now, if $x < 0 < y$, we have $\alpha(x) < 0 < \alpha(y)$. The last case is whenever $x<y\leq0$, yielding $\alpha(x) = -\mu((x,0]) = -\mu((x,y]) - \mu((y,0])$. Since $-\mu((x,y]) \leq 0$, we have that $\alpha(y) \leq \alpha(x)$.
			
			To see that $\alpha$ is continuous, consider the limit of $\alpha(y-\epsilon)=\alpha(y) - \mu((y-\epsilon, y])$. As $\epsilon$ approaches $0$ we get $\alpha(y)$. (This is not true whenever $\mu([y,y])>0$. I cannot figure out how to prove that case.)
			
			For finite real numbers $a$ and $b$ with $a\leq b$, proposition 4.9 shows that $m*((a,b]) = \ell((a,b]) = \alpha(b) - \alpha(a) = \mu((a,b])$. Since such sets $(a,b]$ generate the Borel $\sigma$-algebra, we have that $m*(A) = \mu(A)$ for any $A$ in $\mu$'s domain. Therefore, $\mu$ is the Lebesgue-Stieltjes measure corresponding to $\alpha$
		\end{proof}
	\end{problem}
	
	\begin{problem}{3}
		If $(X, \mathcal{A}, \mu)$ is a measure space, define
		\begin{align*}
			\mu^*(A) = \text{inf}\{\mu(B) : A \subset B, B \in \mathcal{A}\}
		\end{align*}	
		for all subsets $A$ of $X$. Show that $\mu^*$ is an outer measure. Show that each set in $\mathcal{A}$ is $\mu^*$-measurable and $\mu^*$ agrees with the measure $\mu$ on $\mathcal{A}$.
		\begin{proof}
			(I am assuming $\subset$ is inclusive) Since $\mu(\emptyset) = 0$ and $\emptyset \in \mathcal{A}$, $\mu^*(\emptyset) = \mu(\emptyset) = 0$. If $A \subset B \subset X$ then $\{A' : A \subset A', A' \in \mathcal{A}\} \subset \{B' : B \subset B', B' \in \mathcal{A}\}$. Hence, $\text{inf}\{A' : A \subset A', A' \in \mathcal{A}\} \leq \text{inf}\{B' : B \subset B', B' \in \mathcal{A}\}$. Therefore, $\mu^*(A) \leq \mu^*(B)$.
			
			Let $A_1, A_2, A_3, ...$ be a collection of sets in $X$ with $A = \cup_i A_i$. For any $A_1', A_2', ... \in \mathcal{A}$ such that $A_i \subset A_i'$, the set $A' = \cup_i A_i'$ is in $\mathcal(A)$ and $\mu{A'} = \mu(\cup_iA_i') \leq \Sigma_i\mu(A_i')$. Hence, $\text{inf}\{\mu(A') : A \subset A', A' \in \mathcal{A}\} \leq \Sigma_i \text{inf}\{ \mu(A'): A_i \subset A', A' \in \mathcal{A}\}$. So we have $\mu^*(A) \leq \Sigma_i \mu^*(A_i)$. 
		\end{proof}
	\end{problem}
	
	\begin{problem}{5}
		Suppose $m$ is Lebesgue measure. Define $x + A = \{x + y : y \in A\}$ and $cA = \{cy : y \in A\}$ for $x \in \R$ and $c$ a real number. Show that if $A$ is a Lebesgue measurable set, then $m(x + A) = m(A)$ and $m(cA) = |c|m(A)$.
		\begin{proof}
			Multiplying by $c < 0$ results in a reflection of the real number line about the 0 point, which does not affect the Lebesgue measure of any set. So, we can assume WLG that c is positive. The $\ell$ function of the Lebesgue measure is $\ell((a,b])=b - a$. For any collection of intervals $A_i$ in $\mathcal{C}$, $\Sigma \ell(x+A_i) = \Sigma x + b_i - x - a_i = \Sigma b_i - a_i = \Sigma \ell(A_i)$ and $\Sigma \ell(cA_i) = \Sigma cb_i - ca_i = c\Sigma b_i - a_i = c\Sigma \ell(A_i)$. Therefore, $m(x+A) = \inf\{\Sigma \ell(x + A_i) : A_i \in \mathcal{C}, A \subset \cup A_i\} = \inf\{\Sigma \ell(A_i) : A_i \in \mathcal{C}, A \subset \cup A_i\} = m(A)$ and $m(cA) = \inf\{\Sigma \ell(cA_i) : A_i \in \mathcal{C}, A \subset \cup A_i\} = c\inf\{\Sigma \ell(A_i) : A_i \in \mathcal{C}, A \subset \cup A_i\} = cm(A)$
		\end{proof}
	\end{problem}
	
	\begin{problem}{7}
		Suppose $\epsilon \in (0,1)$ and $m$ is Lebesgue measure. Find a measurable set $E \subset [0, 1]$ such that the closure of $E$ is $[0, 1]$ and $m(E) = \epsilon$.
		\begin{proof}
			Consider the set $[\Q \cap (\epsilon,1)] \cup (0,\epsilon)$. The measure of this set is $m([\Q \cap (\epsilon,1)]) + m((0,\epsilon)) = 0 + \epsilon = \epsilon$.
			Since this set contains every rational number in $(0,1)$, its closure is obviously $[0,1]$.
		\end{proof}
	\end{problem}
	
	\begin{problem}{9}
		Let $m$ be Lebesgue measure. Find an example of Lebesgue measurable subsets $A_1, A_2, ...$ of $[0, 1]$ such that $m(A_n) > 0$ for each $n$, $m(A_n \triangle A_m) > 0$ if $n \not = m$, and $m(A_n \cap A_m ) = m(A_n)m(A_m)$ if $n \not= m$.
		\begin{proof}
			Let $A_1$ be the set of measure $0.5$ centered on $0.5$ (ie. $A_1 = (1/4, 3/4)$). For any $A_k$ we define $A_{k+1}$ as the collecting of sets which: 1 contain the endpoints of $A_k$, 2 all have equal measure summing to $\frac{m(A_k)}{2}$, and 3 are each shifted from centering on the endpoints of $A_k$ in such a way that $\frac{m(A_k \cap A_{k+1})}{m(A_{k+1})} = m(A_k)$.
			
			Let $m \not= n$ be such that $m > n$. $m(A_n \triangle A_m)$ is not zero, since $A_n$ contains elements outside of $A_m$. We also have $m(A_n \cap A_)$ //TODO
		\end{proof}
	\end{problem}
	
	\begin{problem}{11}
		Suppose $m$ is Lebesgue measure and $A$ is a Borel
		measurable subset of $\R$ with $m(A) > 0$. Prove that if
		\begin{align*}
			B = \{x - y : x, y \in A\},
		\end{align*}
		then $B$ contains a non-empty open interval centered at the origin.
		This is known as the Steinhaus theorem.
		\begin{proof}
			Given that $A$ is Borel, it can be represented as the union of open intervals. Let $(a,b) \subset A$ be any such interval. With the operations defining $B$, $b-a$ and $a-b$ are the inf and sup elements respectively. Hence, the set in $B$ corresponding to $(a,b)$ is $(a-b, b-a)$, which is obviously centered at $0$
		\end{proof}
	\end{problem}
	
	\begin{problem}{13}
		Let $N$ be the non-measurable set defined in Section
		4.4. Prove that if $A \subset N$ and $A$ is Lebesgue measurable, then $m(A) = 0$.
		\begin{proof}
			Suppose $A$ is any measurable subset of $N$. Then, as in the proof, we have:
			\begin{align*}
				\cup_{q \in [-1,1]\cap\Q}(A+q) \subset [-1,2]
			\end{align*}
			Hence,
			\begin{align*}
				3 \geq \Sigma_{q \in [-1,1],q\in\Q} m(A+q)
			\end{align*}
			Therefore, $m(A)=0$.
		\end{proof}
	\end{problem}
	
	\begin{problem}{15}
		Let $X$ be a set and $\mathcal{A}$ a collection of subsets of $X$ that form an algebra of sets. Suppose $\ell$ is a measure on $A$ such that $\ell(X) < \infty$. Define $\mu^*$ using $\ell$ as in (4.1). Prove that a set $A$ is $\mu^*$-measurable if and only if
		\begin{align*}
			\mu^*(A) = \ell(X) - \mu^*(A^c).
		\end{align*}
		\begin{proof}
			First, consider $m^*(X) = \inf\{\Sigma\ell(A_i) | \cup A_i = X\}$. Setting $A_1, A_2, ...$ to $X, \emptyset, \emptyset$ clearly yields the $\inf$. Hence, $\ell(X) = m^*(X)$.
			
			We can rewrite the given equation as $\ell(X) = \mu^*(A) + \mu^*(A^c)$. Let $E$ be any subset of $X$. Then, subtracting both sides by $\mu^*(E^c)$ yields $\ell(X) - m^*(E^c) = m^*(X) - \mu^*(E^c)$ on the left, which can be expanded to
			\begin{align*}
				\inf\{\Sigma \ell(A_i) | \cup A_i = X\} - \inf\{\Sigma \ell(B_i) | \cup B_i = E^c\}
			\end{align*}
			Subtracting this by $m^*(E)$ yields 
			\begin{align*}
				\inf\{\Sigma \ell(A_i) | \cup A_i = X\} - \inf\{\Sigma \ell(B_i) | \cup B_i = E^c\} - \inf\{\Sigma \ell(C_i) | \cup C_i = E\}
			\end{align*}
			Which can be combined to 
			\begin{align*}
				\inf\{\Sigma \ell(A_i) | \cup A_i = X\} - \inf\{\Sigma \ell(B_i) + \ell(C_i) | \cup B_i = E^c, \cup C_i = E \}
			\end{align*}
			Which is clearly $\inf\{\Sigma \ell(A_i) | \cup A_i = X\} - \inf\{\Sigma \ell(A_i) | \cup A_i = X\} = 0$. Hence, we have $[m^*(X) - m^*(E^c)] - m^*(E) = 0$. This clearly implies that $m^*(X) - m^*(E^c) = m^*(E)$.
			\\
			
			On the right, we have $\mu*(A) + \mu*(A^c) - \mu*(E^c)$, which similarly can be written as 
			\begin{align*}
				\inf\{\Sigma \ell(A_i) | \cup A_i = A\} + \inf\{\Sigma \ell(A_i) | \cup A_i = A^c\} - \inf\{\Sigma \ell(B_i) | \cup B_i = E^c\}
			\end{align*}
			Splitting $\inf\{\Sigma \ell(B_i) | \cup B_i = E^c\}$ into two yields
			\begin{align*}
				 \inf\{\Sigma \ell(A_i) | \cup A_i = A\} + \inf\{\Sigma \ell(A_i) | \cup A_i = A^c\} - \inf\{\Sigma \ell(B_i) | \cup B_i = E^c \cap A\} - \inf\{\Sigma \ell(B_i) | \cup B_i = E^c \cap A^c\}
			\end{align*}
			Rearranging gives us
			\begin{align*}
				[\inf\{\Sigma \ell(A_i) | \cup A_i = A\} - \inf\{\Sigma \ell(B_i) | \cup B_i = E^c \cap A\}] + [\inf\{\Sigma \ell(A_i) | \cup A_i = A^c\} - \inf\{\Sigma \ell(B_i) | \cup B_i = E^c \cap A^c\}]
			\end{align*}
			Combining yields
			\begin{align*}
				[\inf\{\Sigma \ell(A_i) - \ell(B_i) | \cup A_i = A, \cup B_i = E^c \cap A\}] + [\inf\{\Sigma \ell(A_i) - \ell(B_i) | \cup A_i = A^c, \cup B_i = E^c \cap A^c\}]
			\end{align*}
			Which can be rewritten as
			\begin{align*}
				[\inf\{\Sigma \ell(A_i) | \cup A_i = E \cap A\}] + [\inf\{\Sigma \ell(A_i) | \cup A_i = E^c \cap A^c\}]
			\end{align*}
			Which is of course the definition of $m^*(A \cap E) + m^*(A \cap E^c)$. Thus, we have that, $\mu^*(A) = \ell(X) - \mu^*(A^c)$ and for all $E \subset X$, $m^*(E) = m^*(A \cap E) + m^*(A \cap E^c)$, are equivalent statements. The second statement is the definition of $m^*$ measurability. Therefore, $A$ is $m^*$-measurable if and only if	$m^*(A) = \ell(X) - m^*(A^c)$.
		\end{proof}
	\end{problem}
	
	\begin{problem}{17}
		Suppose $A$ is a Lebesgue measurable subset of $\R$ and
		\begin{align*}
			B = \cup_{x\in A} [x - 1, x + 1].
		\end{align*}
		Prove that $B$ is Lebesgue measurable.
		\begin{proof}
			By definition, $m(A) = \inf\{\Sigma (b_i-a_i) | \cup (a_i,b_i) = A\}$ exists. The set $B$ can be defined as //TODO
		\end{proof}
	\end{problem}
\end{document}